\documentclass[twocolumn]{article}
\usepackage{graphicx} % Required for inserting images
\usepackage{xeCJK}
\usepackage{amssymb}
\usepackage{amsthm}
\usepackage{amsmath}
\usepackage{caption}
\usepackage{comment}
\usepackage{subcaption}
\usepackage{enumitem}
\usepackage{multirow}
\usepackage{physics}
\usepackage[separate-uncertainty=true]{siunitx}
\usepackage{multirow}
\usepackage{booktabs}
\usepackage{chemformula}
\usepackage{sectsty}
\usepackage{svg}
\usepackage{url}
\allsectionsfont{\centering}

\setCJKmainfont{NotoSansTC-Regular}
\renewcommand{\baselinestretch}{1.25}
\newcommand{\Rb}{\textrm{Rb}}
\usepackage[a4paper, height=10in, width=8in, hmargin={2cm,0.8in}]{geometry}

\title{112-2 近代物理實驗 Earth Field NMR\\\vspace{1cm}}
\author{第二組 \\ 左:林馳耘 B10202037 中:吉芸萱 B10202036 右:丁安磊 B10202051}
\date{Submitted: Apr. 8, 2024}

\begin{document}

\newcommand{\momega}{~{\rm m}\Omega}
\renewcommand{\figurename}{圖}
\renewcommand{\tablename}{表}
\newcommand{\br}[1]{\left(#1\right)}
\renewcommand{\vb}[1]{\boldsymbol{\mathbf{#1}}} % New vector bold command to adapt for greek letters.
\newcommand\inlineeqno{\refstepcounter{equation}~~ \hspace*{\fill} \mbox{(\theequation)}}

\maketitle

\section{引言與原理}

\subsection{去離子水$T_1$的測量}


\subsection{居禮定律的驗證}
\subsection{磁旋比的測量}
\subsection{一維NMR影像訊號}
\subsection{Spin Echo與去離子水$T_2$的測量}
\subsection{同位素WIP}

\section{回答問題}

\newcommand\question[1]{
  \fbox{\parbox{\dimexpr\linewidth - 2\fboxrule - 2\fboxsep}{#1}}\vspace{2mm}
}

\subsection{問題一}\label{Q1}

\question{
}

\section{結論}
% 本實驗進行了各種對於去離子水 NMR 實驗的測量,我們使用透過調整極化時間以
% 及 180◦pulse 產生的時間測量出 T1、T2、T
% ∗
% 2 ,並符合我們的期待,也透過改變補償
% 電流改變磁場大小計算出質子磁旋比以及透過改變極化電流大小驗證居禮定律。在
% 繪製一維 NMR 影像訊號時,我們發現中心頻率偏移可能和樣品溫度有關,因此我
% 們進行樣品溫度對進動頻率之影響進行量測並且成功解釋中心頻率偏移的主因。


\bibliographystyle{unsrt} % We choose the "plain" reference style
\bibliography{ref.bib} % Entries are in the refs.bib file

% \appendix
% \clearpage

\end{document}