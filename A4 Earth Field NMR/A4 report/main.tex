\documentclass[twocolumn]{article}
\usepackage{graphicx} % Required for inserting images
\usepackage{xeCJK}
\usepackage{amssymb}
\usepackage{amsthm}
\usepackage{amsmath}
\usepackage{caption}
\usepackage{comment}
\usepackage{subcaption}
\usepackage{enumerate}
\usepackage{multirow}
\usepackage{physics}
\usepackage[separate-uncertainty=true]{siunitx}
\usepackage{multirow}
\usepackage{booktabs}
\usepackage{chemformula}
\usepackage{sectsty}
\usepackage{svg}
\usepackage{url}
\allsectionsfont{\centering}

\setCJKmainfont{NotoSansTC-Regular}
\renewcommand{\baselinestretch}{1.25}
\newcommand{\Rb}{\textrm{Rb}}
\usepackage[a4paper, height=10in, width=8in, hmargin={2cm,0.8in}]{geometry}

\title{112-2 近代物理實驗 Earth Field NMR\\\vspace{1cm}}
\author{第二組 \\ 左:林馳耘 B10202037 中:吉芸萱 B10202036 右:丁安磊 B10202051}
\date{Submitted: Apr. 8, 2024}

\begin{document}

\newcommand{\momega}{~{\rm m}\Omega}
\renewcommand{\figurename}{圖}
\renewcommand{\tablename}{表}
\newcommand{\br}[1]{\left(#1\right)}
\renewcommand{\vb}[1]{\boldsymbol{\mathbf{#1}}} % New vector bold command to adapt for greek letters.
\newcommand\inlineeqno{\refstepcounter{equation}~~ \hspace*{\fill} \mbox{(\theequation)}}

\maketitle

\section{引言與原理}

磁矩的進動改變線圈中的磁通量

時變訊號的頻率表示原子核所處的磁場。
訊號的振幅與集體進動的核的淨數量成正比。
集體進動的核的淨數量稱為磁化。

RLC電路原理???
  線圈可以作為放大器和帶通濾波器

  在这个大的极化磁场中“浸泡”原子核的预定时间后,
  将极化电流与线圈断开,线圈中存储的磁能迅速耗散,
  线圈连接到调谐电容器和低噪声放大器。极化磁场消失得如此之快,
  以至于原子核沿着极化场的方向保持极化。随着极化磁场的关闭,
  核磁化在地球磁场中进行,通过样品线圈产生随时间变化的磁通。
  时变磁通产生电动势,电动势在调谐电路的两端产生时变电压。
  该电压由电子器件中的前置放大器和调谐放大器放大。
  进动信号被定向到示波器

\subsection{自旋密度與$T_1, T_2, T_2^*$relaxation}



%自旋態數量分佈、拉莫爾進洞、弛豫、bloch equation、90,180 pulse、溫度


\section{實驗步驟與觀察記錄}

\subsection{線圈校正}
\subsubsection*{實驗步驟}
\subsubsection*{觀察紀錄}
\begin{enumerate}
    \item Trigger,
\end{enumerate}

\subsection{測量$T_1, T_2^*$}
\subsubsection*{實驗步驟}
\subsubsection*{觀察紀錄}

\subsection{驗證居禮定律}
\subsubsection*{實驗步驟}
\subsubsection*{觀察紀錄}

\subsection{測量磁旋比}
\subsubsection*{實驗步驟}
\subsubsection*{觀察紀錄}

\subsection{測量一維 NMR 影像}
\subsubsection*{實驗步驟}
\subsubsection*{觀察紀錄}

\subsection{spin echo 與去離子水 T2 的量測}
\subsubsection*{實驗步驟}
\subsubsection*{觀察紀錄}

\subsection{測量溫度對進動頻率的影響}
\subsubsection*{實驗步驟}
\subsubsection*{觀察紀錄}

\section{結果與討論}












\section{回答問題}

\newcommand\question[1]{
  \fbox{\parbox{\dimexpr\linewidth - 2\fboxrule - 2\fboxsep}{#1}}\vspace{2mm}
}

\subsection{問題一}\label{Q1}

\question{
}

\section{結論}
% 本實驗進行了各種對於去離子水 NMR 實驗的測量,我們使用透過調整極化時間以
% 及 180◦pulse 產生的時間測量出 T1、T2、T
% ∗
% 2 ,並符合我們的期待,也透過改變補償
% 電流改變磁場大小計算出質子磁旋比以及透過改變極化電流大小驗證居禮定律。在
% 繪製一維 NMR 影像訊號時,我們發現中心頻率偏移可能和樣品溫度有關,因此我
% 們進行樣品溫度對進動頻率之影響進行量測並且成功解釋中心頻率偏移的主因。


\bibliographystyle{unsrt} % We choose the "plain" reference style
\bibliography{ref.bib} % Entries are in the refs.bib file

% \appendix
% \clearpage

\end{document}